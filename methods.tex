 
\documentclass[./thesis.tex]{subfiles}
\begin{document}


Quantum chemistry aims at describing the electronic structures of molecular systems.
For atoms of the first 3 rows of the periodic table, relativistic effects can be neglected and the velocity of the nuclei is considered negligible compared to that of the electrons (Born-Oppenheimer approximation). In this context, the model system is a cloud of $N$ electrons and a set of $M$ nuclei considered punctual, immobile charges. It can be described by solving Shrödinger's equation for electrons~:
\begin{equation}
 \widehat{H} \Psi({\bf x}_1,\dots,{\bf x}_N) = E \Psi_n({\bf x}_1,\dots,{\bf x}_N)
\end{equation}
where $\Psi$ is the electronic wave function, $E$ the associated energy, and $\bf x$ . $\widehat H$ is the non-relativistic electronic Hamiltonian operator
\begin{equation}
\widehat{H} = \sum_{i=1}^{N} \Big ( -\frac{1}{2} \Delta_i - \sum_{j=1}^M \frac{Z_j}{|{\bf r}_i - {\bf R}_j|} \Big ) + \sum_{i=1}^{N} \sum_{k>i}^{N} \frac{1}{|{\bf r}_i - {\bf r}_k|}
\end{equation}
${\bf r}_i$ the spatial coordinates of electron $i$, ${\bf R}_j$ and $Z_j$ respectively the spatial coordinate and charge of nuclei $j$.

\section{Slater determinants}

Because of the fermionic nature of electrons, $\Psi$ must satisfy the condition of being anti-symmetric with respect to electron permutation of same-spin electrons.
\begin{equation}
\Psi({\bf r}_1, {\bf r}_2,\dots,{\bf r}_N) = -\Psi({\bf r}_2, {\bf r}_1,\dots,{\bf r}_N)
\end{equation}
Throughout this thesis, we will use the so-called \emph{spin-free} formalism, in which the above condition is achieved by defining two types of electrons, $\alpha$ and $\beta$, and writing the wavefunction as a \emph{Waller-Hartree double determinant},\cite{Pauncz_1989} namely the product of two determinants associated with $\alpha$ and $\beta$ electrons respectively.
\begin{equation}
\begin{array}{c}
 \Psi({\bf r}_1,\dots,{\bf r}_{\Na},{\bf r}_{\Na+1},\dots,{\bf r}_N;
      \alpha_1,\dots,\alpha_{\Na},\beta_{\Na+1},\dots,\beta_N) = \\
D_\alpha({\bf r}_1,\dots,{\bf r}_{\Na}) \times D_\beta({\bf r}_{\Na+1},\dots,{\bf r}_N) = \\
\left|
 \begin{array}{ccc}
 \varphi_1({\bf r}_1) & \dots & \varphi_1({\bf r}_{\Na}) \\
 \vdots               & \ddots &   \vdots             \\
 \varphi_{\Na}({\bf r}_1) & \dots & \varphi_{\Na}({\bf r}_{\Na}) \\
 \end{array}
\right|
\left|
 \begin{array}{ccc}
 \varphi_1({\bf r}_{\Na+1}) & \dots & \varphi_1({\bf r}_{N}) \\
 \vdots               & \ddots &   \vdots             \\
 \varphi_{N_\beta}({\bf r}_{\Na+1}) & \dots & \varphi_{N_\beta}({\bf r}_{N}) \\
 \end{array}
\right| \\ 
\end{array} \\
\label{eq:slater}
\end{equation}
with $\{ \varphi_i \}$ a set of one-electron functions referred to as \emph{molecular orbitals}, or \emph{MO}, typically chosen orthogonal. This type of wavefunction is the spin-free formulation of a \emph{Slater determinant}. We call $N_\alpha$ and $N_\beta \leq N_\alpha$ the number of $\alpha$ and $\beta$ electrons.

Molecular orbitals are typically defined as linear combinations of \emph{atomic orbitals}, or \emph{AO}, here noted $\chi_k$
\begin{equation}
 \varphi_i({\bf r}) = \sum_k C_{ki} \chi_k({\bf r}).
\end{equation}
These functions qualify the used \emph{basis set}, and are usually themselves pre-defined linear combinations of Gaussian functions. This is a restriction put on the form of the wavefunction, therefore it is known as the \emph{finite basis set approximation}.
In the Hartree-Fock method, the wave function is a single Slater determinant, where the $C_{ki}$ coefficients associated with molecular orbitals are optimized so as to minimize the energy, which can be acheived for closed-shell molecules using Roothan equations \cite{Roothaan_1951}. This method, however is missing some important physical effects. For instance, using Eq.\eqref{eq:slater} one can see that in this model opposite-spin electrons are statistically independent (or \emph{uncorrelated}):
\begin{equation}
\begin{split}
\left[ \Psi_\textrm{HF}({\bf r}_1,\dots,{\bf r}_{\Na},{\bf r}_{\Na+1},\dots,{\bf r}_N;
      \alpha_1,\dots,\alpha_{\Na},\beta_{\Na+1},\dots,\beta_N) \right]^2 = \\
\left[ D_\alpha({\bf r}_1,\dots,{\bf r}_{\Na}) \times D_\beta({\bf r}_{\Na+1},\dots,{\bf r}_N) \right]^2 = \\
\left[ D_\alpha({\bf r}_1,\dots,{\bf r}_{\Na}) \right]^2 \times \left[ D_\beta({\bf r}_{\Na+1},\dots,{\bf r}_N) \right]^2.
\end{split}
\end{equation}

\section{Electron correlation}
Even though typically, only one or a few $\ket{\Psi_n}$ associated with lowest eigenvalues are of interest, the Shrödinger's equation cannot be solved exactly (except for a few small systems), and more or less drastic approximations need to be used. 

Electron correlation is defined as\cite{Lowdin_1959}
\begin{equation}
\Ecor = E_\text{exact} - E_\text{HF}
\end{equation}
where $E_\text{HF}$ is the \emph{Hartree-Fock limit}, i.e. the limit to which the Hartree-Fock energy converges when the size of the basis set increases.

To include electron correlation effects, $\Psi$ may be expanded in
$\{\ket D_i \}$, the set of all the possible Slater determinants that can be built by putting $N_\alpha$ electrons in $M$ orbitals and $N_\beta$ electrons in $M$ orbitals.
The eigenvectors of $\widehat{H}$ are consequently expressed as a linear combination of Slater determinants 
\begin{equation}
\ket{\Psi_n} = \sum_{i} c_i^n \ket{D_i}.
\end{equation}
Solving the eigenequation for state $n$ in this basis is referred to as \emph{Full Configurations Interaction (FCI)} and yields a solution for Schrödinger's equation that is exact for the given atomic basis set.
Matrix elements of $\widehat H$ are integrals over $3N$ dimensions. However, given its bi-electronic nature, Slater determinants that differ by more than 2 electrons are known to have null matrix elements, and the others can be expressed as sums of integrals over \alert{3 or 6} dimensions, which can be computed for a reasonable cost. These simplifications are known as \emph{Slater-Condon rules} \alert{a re-checker imperativement}

% \Hij[\hat h]{\varphi_i \varphi_j}{\varphi_i \varphi_j} - \Hij[\hat h]{\varphi_i \varphi_j}{\varphi_j \varphi_i} 
\begin{align}
\Hij{D}{D} = \frac{1}{2} \sum_{i=1}^N \sum_{j=1}^N \Big [ (ii|jj) - (ij|ij) \Big ]      \\
\Hij{D}{D_p^r} = \sum_{i=1}^N \Big [ (pr|ii) - (pi|ri) \Big ]      \\
\Hij{D}{D_{pq}^{rs}} = (pr|qs) - (ps|qr)
\end{align}

with
%\Hij[\hat h]{\varphi_i \varphi_j}{\varphi_k \varphi_l} = \int d{\bf r}_1 \int d{\bf r}_2 \; \varphi^*_i({\bf r}_1)\varphi^*_j({\bf r}_2) \frac{1}{|{\bf r}_1 - {\bf r}_2|} \varphi_k({\bf r}_1)\varphi_l({\bf r}_2)
\begin{equation}
(ij|kl) = \int d{\bf r}_1 \int d{\bf r}_2 \; \varphi^*_i({\bf r}_1)\varphi_j({\bf r}_1) \frac{1}{|{\bf r}_1 - {\bf r}_2|} \varphi^*_k({\bf r}_2)\varphi_l({\bf r}_2)
\end{equation}

and $\ket {D_{pq}^{rs}}$ a determinant obtained from $\ket D$ by moving electrons from \alert{orbitals} $p,q$ to $r,s$.

In spite of this, computing these integrals is a potential bottleneck. Indeed it is possible to analitycally compute an integral $(pq|rs)$ with $p,q,r,s$ atomic orbitals, but molecular orbitals are desired. The cost of computing a single one scales as $N^4$.

\begin{equation}
(ik|jl) = \sum_{pqrs} C_{pi} C_{qk} C_{rj} C_{sl} (pq|rs)
\end{equation}

A naive computation of all integrals in the MO basis would cost ${\cal O}^8$. Fortunately, computing all can be scaled down to ${\cal O}^5$ \cite{Bender_1972} :
\begin{eqnarray}
(iq\vert rs) & = & \sum_{p} C_{pi} (pq|rs) \nonumber \\ 
(iq\vert js) & = & \sum_{r} C_{rj} (iq|rs) \;\; \text{half-transformed??? integrals} \\
(ik\vert js) & = & \sum_{q} C_{qk} (iq|js) \nonumber \\ 
(ik\vert jl) & = & \sum_{s} C_{sl} (ik|js)  \;\; \text{fully transformed integrals}
\end{eqnarray}

In addition to being very costly, this step is hard to parallelize in distributed way, because all half-transformed integrals are required to compute fully transformed integrals, which implies an "all-to-all" communication of the full set of data.

FCI is only acheivable for very small systems, given that $\NFCI$ the size of the determinants basis scales factorially with the number of molecular orbitals
\begin{equation}
\NFCI = \frac{M!}{N_\alpha ! (M-N_\alpha)!} \times \frac{M!}{N_\beta ! (M-N_\beta)!}
\end{equation}
Post-Hartree-Fock methods are trying to circumvent this problem, and therefore are essentially approximations of FCI.


%Different approdavidsonaches exist.


\alert{ Il faut mettre ici l'expression des elements de matrice de H dans la
base des determinants.  Il faut aussi dire que pour Hartree-Fock, en utilisant
les equations de Roothaan on peut resoudre le probleme dans la base des AOs.
Par contre, le probleme est beaucoup complique dans une base de fonctions a
N-electrons car chaque element de matrice doit faire intervenir des integrales
a N electrons.  Si les fonctions de base a N-electrons sont des determinants
fabriques a partir de fonctions a 1 electron orthonormales, telles que les
determinants de Slater, le calcul des elements de matrice de H devient beaucoup
plus simple grace aux regles de Slater-Condon qui permettent de ne faire
intervenir que des integrales a 2 electrons. Il faut donc introduire les regles
de Slater-Condon.  Il faut aussi introduire ici la four-index transformation,
qui est un des bottlenecks des methodes post-HF a cause de son scaling en
$\order{N^5}$.  } 

\section{Variational post-Hartree-Fock methods}

In variational methods, Schrödinger's equation is solved in an $N$-electron basis.
Generally speaking, solving Schrödinger's equation in a basis of Slater determinants is called \emph{Configuration Interaction (CI)}, the FCI being the particular case where the whole $\{\ket {D_i} \}$ set is used.
The general idea of these methods is to select \textit{a priori} a relevant subset of Slater determinants in which the CI problem will be solved.

One usual approach is to effectively reduce the number of molecular orbitals, by performing a FCI in a reduced set of orbitals and freezing the occupation status for the others. This is referred to as \emph{Complete Active Space Configuration Interaction (CAS-CI)}. Choosing the CAS orbitals can require some expertise. The CAS-SCF method minimizes the energy by both performing a CAS-CI and optimizing the molecular orbitals.

Another usual approach is to select determinants according to their excitation degree --- by how many occupied orbitals they differ --- with respect to some reference. If this reference is the Hartree-Fock determinant and only single and double excitations are considered, the method is known as \emph{Configuration Interaction with Single and Double excitations (CISD)}. Alternatively, the reference can be a CAS, in which case it is known as \emph{Multi-Reference Configuration Interaction (MR-CI)}.

Regardless of the method, integrals involving non-frozen orbitals need to be computed in order to diagonalize $\widehat{H}$. Therefore CI methods' cost is at least $\order{N^5}$, due to the four-index transformation.
In addition, the cost for fully diagonalizing $\widehat{H}$ is $\order{\Ndet^3}$ with $\Ndet$ the number of determinants in the considered subspace, which can be up to a few billions. This usually not feasible, but only the few eigenvectors associated with lowest eigenvalues are typically of interest, so iterative methods can be used. The Quantum Package uses the \emph{Davidson diagonalization} originally developped by Ernest R. Davidson\cite{Davidson_1975} specifically for CI methods. 

\alert{On voit que pour faire un CI, on a besoin de toutes les integrales
bielectroniques. Donc ca coute au moins $\order{N^5}$.  Il faut diagonaliser
$H$ dans la base des determinants, et il y en a souvent un tres grand nombre
(plusieurs millions, voire milliards). Donc je pense qu'il faut introduire
Davidson ici. Mettre dans la partie 'determiant driven' l'implementations qu'on
a dans le QP, et tes reflexions sur Davidson en dans les perspectives a la fin.
}

\section{Perturbative methods}


\alert{Pioche un peu chez Manus pour introduire MP2 et Epstein Nesbet.}



\alert{On voit que pour MP2, on n'a pas besoin de toutes les integrales
bielectroniques : les $\braket{vv}{vv}$ ne sont pas necessaires. Donc ca coute
forcement moins cher que le CI.}

\section{Selected CI methods}

These methods rely on the same principle as the usual CI approaches, except that determinants aren't chosen \textit{a priori} based on an occupation or excitation criterion, but selected \emph{on the fly} among the entire set of determinant based on their estimated contribution to the FCI wavefunction. Conventional CI methods can be seen as an exact resolution of Schrödinger's equation for a complete, well-defined set of determinants (and for a given atomic basis set), while selected CI methods are more of a truncation of the FCI.
The main advantages of these methods compared to the more conventional \textit{a priori} selected ones, are that since the most relevant determinants are considered, they will typically yield a more accurate description of physical phenomena, and a much lower energy for an equivalent number of determinants.
It has been noticed long ago that, even inside a predefined subspace of determinants, only a small number significantly contributes to the wavefunction. Therefore, an \emph{on the fly} selection of determinants is a rather natural idea that has been proposed from the late 60s by Bender and Davidson\cite{Bender_1969} as well as Whitten and Hackmeyer\cite{Whitten_1969}.

The approach we are using is based on the \emph{Configuration Interaction using a Perturbative Selection (CIPSI)} developped by Huron, Rancurel and Malrieu \cite{Huron_1973}, that iteratively selects determinants by using a perturbative criterion, and computes an estimation to the electronic correlation missing from the variational energy.
Iteration $n$ of CIPSI can be described like so:

\begin{enumerate}
\item

The variational function $\ket {\Psi^{(n)}}$ is defined over a set of determinants $  \mathcal{D}^{(n)}$ in which we diagonalize $\widehat{H}$
\begin{equation}
\ket{\Psi^{(n)}} = \sum_{S \in \mathcal{D}^{(n)}} c_S^{(n)} \ket{S}
\end{equation}

\item
For all $\kalpha \notin \mathcal{D}^{(n)}$, we compute a perturbative contribution
\begin{equation}
e_\alpha = \frac{\Hij{\Psi^{(n)}}{\alpha}^2}{E^{(n)} - H_{\alpha \alpha}}
\end{equation}
$E^{(n)}$ is an energy that depends on the pertubation theory being used. In our case, the Epstein-Nesbet energy is used.

\item
We build $\{ \alpha^\star \}^{(n)}$ the set of $\kalpha$ of highest contributions, and add them to the wavefunction
\begin{equation}
\mathcal{D}^{(n+1)} = \mathcal{D}^{(n)} \cup \{ \alpha^\star \}
\end{equation}

\item
Exit on some criterion. Note that $\EFCI$ energy can be estimated
\begin{equation}
\EFCI \approx E^{(n)}_0 + E_{PT2}^{(n)}
\end{equation}


with $E^n_0$ the variational energy of $\ket {\Psi^{(n)}}$ and
\begin{equation}
E_{PT2}^{(n)} = \sum_{\alpha \in \{\alpha \}^{(n)}} e_\alpha
\end{equation}
with $\{ \alpha \}^{(n)}$ the set of all $\kalpha$ considered at current iteration

\end{enumerate}

There is however a computational downside. In \textit{a priori} selected methods, the rule by which determinants are selected is known a priori, and therefore, one can map a particular determinant to some row or column index.\cite{Knowles_1984} As a consequence, it can be systematically determined to which matrix element of $\widehat{H}$ a two-electron integral contributes. This allows for the implementation of so-called \emph{intergral-driven} methods, that work essentially by iterating over integrals.
On the contrary, in selected methods an explicit list has to be maintained, and there is no immediate way to know whether a determinant has been selected, or what its index is. Consequently, so-called \emph{determinant-driven} approaches will be used, in which iteration is done over determinants rather than integrals. This can be a lot more expensive, since the number of determinants is typically much larger than the number of integrals. The former scales as $\order{N!}$ while the latter scales as $\order{N^4}$ with the number of MOs.
Furthermore, determinant-driven methods require an effective way to compare determinants in order to extract the corresponding excitation operators, and a way to rapidly fetch the associated integrals.

Because of this high computational cost, approximations have been proposed \cite{Evangelisti_1983},  . And recently, the \emph{Heat-Bath Configuration Interaction (HCI)} algorithm has taken farther the idea of a more approximate but extremely cheap selection\cite{Holmes_2016, Sharma_2017}. Compared to CIPSI, the selection criterion is simplified to
\begin{equation}
e^{HCI}_\alpha = \text{max} \big  {(|} c_i \Hij{D_i}{\alpha} \big {|)}
\end{equation}
with $D_i$ a determinant from the variational wavefunction and $c_i$ its associated coefficient.

\section{Monte-Carlo ?}
FCI-QMC is an alternate approach to selection recently proposed in 2009 by Alavi \textit{et al.}\cite{Booth_2009,Cleland_2010}, where walkers move from one determinant to another with a probability that is function of the associated matrix element. \alert{The average population of walkers on a determinant converges to its coefficient in the FCI wavefuction?}.

\alert{verifier citation, j'ai pas Coe mais Greer?} A more ``bruteforce'' approach at stochastical selection is \emph{Monte-Carlo CI (MCCI)} \cite{Greer_1995}, where determinants are randomly added to the variational wavefunction. After diagonalization, the determinants of smaller coefficient are removed, and new random determinants are added.




\end{document}
