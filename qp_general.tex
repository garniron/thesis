


\section{Fortran}
It's mostly based on Fortran code, a low-level, high-efficiency language popular among theoretical chemists. This facilitates use and compatibility with previously developped quantum chemsitry code, and guarantees good performances. It's natively linked to the popular LAPACK linear algebra package

\section{IRPF90}
It uses IRPF90, a metalanguage(?) build on top of Fortran, that allows a dataflow-like programming style.
IRPF90 allows create so-called "providers", which are essentially nodes of dataflow strucutres that can be used in a nearly transparent way by the user. It is fully compatible(****) with normal fotran code. Code examples will be given later... I guess... in chapter something... when it will be written.
(...ou alors exemple ici)


\section{EZFIO}


avoids cumbersome IO / serializing / etc... to an extent

EZFIO.cfg

exemple

\section{Modular structure}

It has a modular structure that allows to easily implement new methods and re-use pre-existing code.
Users can create so-called "module" for their algorithms, effectively isolating them from pre-existing code. A module can import code from other modules, but is compiled into a stand-alone executable(****). This ensures easier sharing of developped algorithms.
Typically a module is run on a specific 
(ptet exemple avec generator\_CAS versus generator\_full, mais on a pas encore parler de determinant-driven )
PSI is filtered with different function and supplied to the module using cas\_sd ( cas\_sd, mrcc ) or perturbation ( pt2 stoch )??

automatic creation of interdependent modules, with theirs own inputs and outputs ( EZFIO.cfg, discussed later ) \\
******** (ptet exemple avec generator\_CAS versus generator\_full, mais on a pas encore parler de determinant-driven >\_< PSI is filtered with different function and supplied to the module using cas\_sd ( cas\_sd, mrcc ) or full\_ci ( pt2 stoch ) )?? \\

\section{ZMQ}
More recently, it allows for easy creation of massively parallel, task-based algorithms.
Using the very flexible ZMQ library, a framework has been set up to allow users to define elementary tasks that are automatically distributed to remote nodes via TCP. Each task's result is sent back to a central, user-defined "collector". The main advantage of using TCP communication rather than the "usual" MPI implementation, is asynchronicity, flexibility in adding/removing slave nodes - and thus fault-tolerance, and the ability to use nodes through internet.


standard choice would be MPI \\
ease of use \\
flexibility in adding/removing nodes \\
... thus fault tolerance \\
asynchronicity \\
can act as a layer above MPI \\
\section{High-level task distribution - OCaml}
task "administration" is hard with low-level language such as Fortran \\
functional languages are more bug-proof \\

task "administration" is hard with low-level language such as Fortran ; but it has to be moved away from the user \\
ask him to only define a list of tasks ( as strings ), function(task) and collector(function(task)) \\
transparently distribute these tasks to remote nodes \\
template for boiler plate \\
