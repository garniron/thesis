 
\section{Generating a subspace of determinants}
\alert{
duplicates are avoided by checking connection to the past \\
******** (ptet exemple avec generator\_CAS versus generator\_full PSI is filtered with different function and supplied to the module using cas\_sd ( cas\_sd, mrcc ) or full\_ci ( pt2 stoch ) )?? \\
================
}

Generally speaking, a subspace of determinants is defined by a reference and a class of excitation applied to it.
\alert{
Because of the determinant-driven approach, it may not be possible to store all the determinants of the considered subspace. \textbf{Pourquoi c'est lie au determinant-driven?}
}



\paragraph{Reference space}
The \QP allows to define either a determinants of a CAS (Complete Active Space) or the full-CI space as the reference.\footnote{Although the full-CI space is a particular CAS, it is treated differently at the code level.}

\paragraph{Excitation class}
The excitations to be applied to determinants of the reference space are defined by sets of orbitals describing:
\begin{itemize}
\item
Single excitations
\item
The first excitation of a double excitation
\item
The second excitation of a double excitation
\end{itemize}
For each set category, there is one set for the allowed holes and one set for the allowed particles.
This defines the rules used to generate determinants of the external space from the reference determinants.

Typically, to define a CAS-SD (CAS augmented with all single and double excitations applied to all the determinants of the CAS), one defines:
\begin{description}
\item [Core orbitals:]
Orbitals always doubly occupied. They are not part of any of the sets.
\item [Inactive orbitals:]
Orbitals in which one or two holes may be created. They are part of the all the \emph{hole} sets.
\item [Active orbitals:]
Orbitals defining the CAS. They are part of all the sets.
\item [Virtual orbitals:]
Orbitals in which one or two particles may be created. They are part of the all the \emph{particle} sets.
\item [Deleted orbitals:]
Orbitals removed from the calculation, always unoccupied. They are not part of any set.
\end{description}


 As will be seen later, determinants are selected iteratively. The general procedure is shown 



\begin{algorithm}
        \caption{GENERAL\_SELECTION}    
        \label{alg:GENERAL_SELECTION}   
        $\ket{\Psi} \gets \ket{\HF}$\;
        Build the sets $\set{H}_0, \set{H}_1, \set{H}_2, \set{P}_0, \set{P}_1, \set{P}_2$ \;
        $\set{T}$: the set of excitations allowed on the reference based on $\set{H}_0, \set{H}_1, \set{H}_2, \set{P}_0, \set{P}_1, \set{P}_2$ \;
        
        \While{some condition}{
                $\set{D}$: the set of determinants in $\ket{\Psi}$ \;
                $\set{D}' \gets \{ \}$ \;
                \For{$i \gets 1, \Ndet$}{
                        \ForAll{$\hat{T}_j \in \set{T}$}{
                                $\kalpha \gets \ordering \hat{T}_j \ket {D_i}$ \;
                \If{$\kalpha \neq 0$}{
                /* Check if $\kalpha$ was not already generated before */ \;
                \textit{found} $\gets$ False \;
                \For{$l \gets 1, i-1$}{
                        \ForAll{$\hat{T}_k \in \set{T}$}{
                                $\textit{found} \gets \ior{\textit{found}}
            {(\ordering \hat{T}_k \ket{D_{l}} \neq \kalpha)}$ \;
                                }
                                }
                \If{$\iand{(\neg \textit{found})}{\textit{Selection\_criterion}}$ }{
                $\set{D}' \gets \set{D}' \cup \kalpha $ \;                      
                }
                                }
                        }
                }
                do some selection in $\set{D}'$ \;
                add selected determinants to $\set{D}$ \;
                diagonalize $\hat{H}$ in $\set{D}$ to obtain $\ket{\Psi}$ \;
        }
\end{algorithm}
