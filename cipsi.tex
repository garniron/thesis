\documentclass[./thesis.tex]{subfiles}

\newcommand{\Gpqrs}{\ket{G^{rs}_{pq}}}
\begin{document}
\label{chap:CIPSI}


\section{The base algorithm}
\alert{"pas d'excitation" $\implies \hat T = 1$ ?}\\
My initial and most important work has been the improvement of the CIPSI algorithm present in the \QP, that had been implemented by my predecessor\cite{giner:tel-01077016}. As was briefly described in the previous chapter, it is an \emph{on the fly} iterative selection algorithm, where determinants are added to the variational wavefunction according to a perturbative criterion. Because it gathers a large amount of information, this CIPSI algorithm has been the basis for other subsequent works presented in the next chapters.

Iteration $n$ of CIPSI can be described like so:

\begin{enumerate}
\item

The variational function $\ket {\Psi^{(n)}}$ is defined over a set of determinants $  \mathcal{D}^{(n)}$ in which we diagonalize $\widehat{H}$
\begin{equation}
\ket{\Psi^{(n)}} = \sum_{S \in \mathcal{D}^{(n)}} c_S^{(n)} \ket{S}
\end{equation}
The determinants in $\mathcal{D}^{(n)}$ will be characterized as \emph{internal}.

\item
For all \emph{external} determinants $\kalpha \notin \mathcal{D}^{(n)}$, we compute a perturbative contribution
\begin{equation}
e_\alpha = \frac{\Hij{\Psi^{(n)}}{\alpha}^2}{E^{(n)} - H_{\alpha \alpha}}.
\end{equation}
%$E^{(n)}$ is an energy that depends on the pertubation theory being used. In our case, the Epstein-Nesbet energy is used.
As we use Epstein-Nesbet perturbation theory, $E^{(n)}=\Evar^{(n)}$ is the variational energy of the wave  function at the current iteration (note that another perturbation theory could be used here).

\item
From $\{ \alpha \}^{(n)}$ the set of all $\kalpha$ considered at the current iteration, we extract $\{ \alpha^\star \}^{(n)}$ the set of $\kalpha$ of largest contributions $e_\alpha$, and add them to the wavefunction
\begin{equation}
\mathcal{D}^{(n+1)} = \mathcal{D}^{(n)} \cup \{ \alpha^\star \}
\end{equation}

\item
The FCI energy $\EFCI$ can be estimated
\begin{equation}
\EFCI \approx \Evar^{(n)} + \EPT^{(n)}
\end{equation}
with $\Evar^{(n)}$ the variational energy of $\ket {\Psi^{(n)}}$ and
\begin{equation}
\EPT^{(n)} = \sum_{\alpha \in \{\alpha \}^{(n)}} e_\alpha
\end{equation}

\item
Go to iteration $n+1$, or exit on some criterion (number of determinants in the wavefunction, low $\EPT^{(n)}$, \dots).

\end{enumerate}



As can be seen, CIPSI involves the creation of an external space and a precise knowledge of how it interacts with the internal space. Algorithmically speaking, we will need to enumerate all connections between all internal and all external determinants.
There are, perhaps schematically, two ways to do this :

\begin{itemize}
\item
``external to internal'', looping over all possible $\kalpha$ and computing $e_\alpha$.
\item
``internal to external'', looping over all internal determinants $\ket G$ and all single or double excitations $\hat T$, creating $\kalpha = \hat T \ket G$, then incrementing $\tilde e_{\alpha \notin \Psi}$ by $\Hij{G}{\alpha}$. Finally get $e_{\alpha \notin \Psi} = \frac{\tilde e_\alpha^2}{E^{(n)} - H_{\alpha \alpha}}$.
\end{itemize}

While the second approach sounds more efficient, it has the obvious issue of requiring all $e_\alpha$ to be stored in memory simultaneously. Unfortunately this is usually not feasible, since their number scales as $\order{\Ndet \times \mathcal{O}_{virt}^2 \times \mathcal{O}_{occ}^2}$, with $\mathcal{O}_{occ}$ the number of occupied orbitals and $\mathcal{O}_{virt}$ the number of virtual orbitals.
While the first approach is simpler, it begs the question of how to generate all possible $\kalpha$ not simultaneously and with no duplicates. 


\section{Approximations}

Given the qualitative nature of this procedure --- each $\kalpha$ is either selected or not --- it is possible to save a vast amount of computation with minimal approximations. These were present in the original implementation and retained in the new one.

Both the former and newer version of CIPSI generate the external space in an ``internal to external'' way, that is, by applying single and double excitations to internal determinants ; a determinant used for creation of $\kalpha$ is referred to as a \emph{generator}.

Ensuring each $\epsilon(\alpha)$ is considered only once, is done by checking all newly generated $\kalpha$ for connection (at most double excitation) to a determinant $\ket {D_I}$ previously used as a generator ; if a connection $\hat T$ is found, it means $\kalpha$ was already generated from $\ket {D_I}$ as $\hat T \ket {D_I}$.

Determinants $D_I \in \Psi$ are sorted by decreasing absolute values of $c_I$. Two approximations are made :

\begin{itemize}
\item
The first approximation restricts the set $\{\kalpha\}$. It is very unlikely $\ket \alpha$ will be selected if it's not connected to any $\ket D$ with a large coefficient. Therefore, it is possible to only consider the determinants of larger coefficient as generators. We choose a number of generators $\Ngen$ and only consider $D_{I \leq \Ngen}$ as generators. In practice we set $\Ngen$ according to a norm threshold $n_g$, picking $\Ngen$ as the highest value fulfilling
\begin{equation}
\sum_{I \leq \Ngen} c_I^2 \leq n_g.
\end{equation}
Typically, $n_g=0.99$ is used in the calculations.
\item The second approximation reduces the cost of $\epsilon_\alpha$.
We do not need extremely accurate values for $\epsilon(\alpha)$ as small differences are unlikely to substantially change the subset of the largest ones.
So  connections to $\ket {D_I}$ with small coefficients $c_I$ can be neglected
in the expression of $\epsilon_\alpha$.
This approximation is achieved in a similar way by defining a threshold $n_s$
on the norm of the wavefunction, and $N_{sel} \geq \Ngen$ a number of so-called
\emph{selectors}. We \alert{approximate approximating - correct mais bizarre?}
\begin{equation}
  \Hij{\Psi}{\alpha} \approx \sum_{i \leq N_{sel}} c_i \Hij{D_i}{\alpha}.
\end{equation}
Typically, we use $n_s = 0.999$.

\end{itemize}

Note that generator determinants are a subset of selector determinants. See figure \ref{fig:generators_selectors}.


\begin{figure}[h!]
        
        \begin{center}
                \includegraphics[width=0.4\columnwidth]{figures/cipsi/selexemple2}
                \caption{$\ket \Psi$ is sorted by decreasing ${c_I}^2$; generator and selector subsets are defined.}
                \label{fig:generators_selectors}
        \end{center}
\end{figure}



\section{Initial implementation}

Originally, the Quantum Package generated the external space in a ``internal to external'' way, by applying all excitations on all determinants ; but the computation of $e_\alpha$ itself was a straightforward ``external to internal'', computing a single $e_\alpha$ at a time, avoiding the problem of keeping track of all $\epsilon(\alpha)$ simultaneously.

While this implementation is obsolete, since it is relatively simple, it is briefly presented for pedagogical reasons. A slightly more detailed algorithmic version is shown as algorithm \ref{alg:cipsi_manu}.

\begin{enumerate}
\item
$\ket {D_I}$ are sorted with decreasing ${c_I}^2$.
\item
loop over generators $\ket G$
\item
generate all singly and doubly excited determinants connected to $\ket G$
\item
from this set, discard those that appear in $\Psi$. This is now a set of $\ket \alpha$
\item
from this set, discard those that have already been generated before, i.e. those connected to $D_K$ with $K<I$. This is now a set of unique $\ket \alpha$.
\item
compute $\epsilon(\alpha) = \frac{\langle \alpha|H|\Psi\rangle^2}{\Delta E_\alpha}$ for those new $\ket \alpha$.
\end{enumerate}


\begin{algorithm}
        \label{alg:cipsi_manu}
        \caption{Simple CIPSI}
                \KwData{ $\ket \Psi$ }
                \KwResult{ Guarantees all $\epsilon(\alpha)$ are computed a single time. }
                sort $\ket {D_I}$ by decreasing ${c_i}^2$ \;
                \For {$g \gets 1, \Ngen$}{
                \tcc{apply all double excitations on $|D_g \rangle$}
                \ForAll {$\ket \alpha$ ; $\langle D_g | H | \alpha \rangle \neq 0$}{
                \For{$p \gets 1, g-1$}{
                        \If{$\ket \alpha$ connected to $\ket {D_p}$}{
                                \tcc{$\ket \alpha$ has already been generated by $\ket {D_p}$}
                                discard this $\ket \alpha$ \;
                        }
                }
                search $\kalpha$ in $D_{\Nsel+1},\ldots,D_{\Ndet}$ \;
                \If{$\kalpha$ was found}{
                        discard this $\kalpha$
                }
                 $R \gets 0$ \;
                \For{$s \gets g, \Nsel$}{
                        $R \gets R + \Hij{D_s}{\alpha}$ \;      
                        \tcc{$\ket {D_s}==\kalpha$ is noticed when computing $\Hij{D_s}{\alpha}$}       
                        \If{$\ket {D_s}==\kalpha$}{
                                \tcc{$\kalpha \in \Psi$}
                                discard this $\ket \alpha$
                        }
                }
                 assert $R == \Hij{\Psi}{\alpha}$  \;
                 $\epsilon(\alpha) = \frac{R^2}{\Delta E_\alpha}$
                }
                }
\end{algorithm}


\section{Principle of the new algorithm}

The current approach is intermediate between computing $\epsilon(\alpha)$ one by one, and keeping track of all of them at the same time.
It creates a subset, or \emph{batch} of extermal determinants small enough to fit into memory, and importantly, that isn't arbitrary.
A batch is defined by a doubly ionized generator
\begin{equation}
\ket {G_{pq}} = a_p a_q \ket G.
\end{equation}
Determinants contained in the $\ket {G_{pq}}$ batch , some of which may be unique $\kalpha$, can be systematically defined by two indices $r$ and $s$ with
\begin{equation}
\ordering a^\dagger_r a^\dagger_s a_p a_q  \ket G = \Gpqrs.
\end{equation}

Essentially, determinants in a batch are defined by their difference to $G_{pq}$. Therefore, comparing $G_{pq}$ to a selector determinant allows to systematically determine which $\kalpha$ of the batch it will connect to, and by what excitation. Additional filtering mechanisms are set up to avoid considering selectors that do not interact with the current batch. Those will be made explicit later on. Comparing figures \ref{fig:old_cipsi} and \ref{fig:new_cipsi} hints the differences between the former an newer algorithm. Note that because generators are a subset of selectors, a particular $\kalpha$ generated from $\ket {D_g}$ must be checked for connection to all selectors either as generators or as selectors.

\begin{itemize}
\item
$\ket {D_{1 \le I < g}}$ as generators to check if $\kalpha$ has been previously generated
\item
$\ket {D_{g \le I \le \Nsel}}$ as selectors to compute $\Hij{\Psi}{\alpha}$.
\end{itemize}


\begin{figure}[h!]
        \begin{center}
                \includegraphics[width=0.7\columnwidth]{figures/cipsi/old_cipsi}
                \caption{Original CIPSI schematic representation, some details omitted}
                \label{fig:old_cipsi}
        \end{center}
\end{figure}


\begin{figure}[h!]
        \begin{center}
                \includegraphics[width=0.8\columnwidth]{figures/cipsi/new_cipsi}
                \caption{New CIPSI schematic representation, some details omitted. 
\alert{ Fautes d'orthographe : pas de 'e' a implicitly et 'external determiNants'} }
                \label{fig:new_cipsi}
        \end{center}
\end{figure}

\subsection{Unfiltered algorithm}

Filtering of selectors is a somewhat natural idea that was actually implemented before the batch approach. It however can easily be understood as something added ``on top'' of it, so it will be detailed in the next section and ignored in this one.



\begin{enumerate}
%\setlength{\itemindent}{2em}
\item
Iterate over $\ket {D_{I \leq \Ngen}} = \ket G$
\item
Iterate over all possible $a_p a_q \ket G = \ket {G_{pq}}$
\item
allocate a zero-initialized matrix $P(G_{pq})$ indexed by $r$ and $s$. Each cell is associated with $\ordering a^\dagger_r a^\dagger_s a_p a_q  \ket G = \Gpqrs$. Some cells will be tagged as not being associated with a unique $\ket \alpha$, but either one of :
\begin{itemize}
\item
a determinant already present in the wavefunction
\item
a non-existing determinant, i.e. $\Gpqrs = 0$
\item
a non-unique $\ket \alpha$ (either a double excitation of a previous generator, or a single excitation of the current one)
\end{itemize}

\item
Since two electrons cannot occupy the same spinorbital, tag cells where $r$ or $s$ is occupied in $\ket {G_{pq}}$.
\item
Apply single excitation tagging. This ensures single excitations of $\ket G$ are generated exactly once. It is described in section \ref{single_tagging}.
\item
\textbf{selector loop}: Iterate over $\ket {D_J} = \ket S$ 
\item
Determine whether there is a pair $(r,s)$ so that $\ket S = \ket {G_{pq}^{rs}}$. In other words, look for $\ket S$ in the current batch. If there is, tag the corresponding cell, $\Gpqrs \in \Psi$.
\item
Determine $(r,s)$ pairs so that $\ket {G_{pq}^{rs}}$ is connected to $\ket S$
\item
If $J<I$, tag the corresponding cells ; $\Gpqrs$ has already been generated by $\ket {D_J}$.
\item
If $J \geq I$, increment all untagged $P_{rs}(G_{pq})$ matrix elements by $\Delta P_{rs}(G_{pq}) = c_J \Hij{S}{G^{rs}_{pq}}$. Note that the excitation operator $\hat{T}$ so that $\ket S=\hat{T} \Gpqrs$, useful for computing the associated matrix element, can be determined at the same time as the $(r,s)$ pair.
\item
End \textbf{selector loop}. All untagged cells are guaranteed to be associated with a unique $\ket \alpha$ and $P_{rs}(G_{pq}) = \langle \Psi |H|G^{rs}_{pq} \rangle$. $\epsilon(\alpha)$'s for the current batch can be computed  \\

\begin{equation}
\epsilon(\ket {G_{pq}^{rs}}) = \frac{P_{rs}(G_{pq})^2}{\Delta E_{\Gpqrs}}
\end{equation}
\item
End of other loops. All $\epsilon(\alpha)$ have been computed a single time.

\end{enumerate}


\subsection{Tagging}

Tagged cells are simply tracked using a boolean matrix $B(G_{pq})$ with $B_{rs}(G_{pq})$ keeping the tag status of $\Gpqrs$, defaulting to $\mathtt{FALSE}$. 
In some cases, full columns/rows are to be tagged. Keeping track of fully tagged rows or columns is useful for performance purpose, as it allows to shortcut some loops. A simple way to do it, is to add an extra column and an extra row of index $0$ to $B$ ; $B_{0s}(G_{pq}) = \mathtt{TRUE}\,$ means the whole $s$ column is tagged, $B_{r0}(G_{pq}) = \mathtt{TRUE}\,$ means the whole $s$ line is tagged. The actual tag status of $\Gpqrs$ becomes
\begin{equation}
B_{r0}(G_{pq}) \vee B_{0s}(G_{pq}) \vee B_{rs}(G_{pq}).
\end{equation}
While significant, this optimization is fairly simple to set up and use, so for simplification purpose, it will be ignored in the text.


\subsection{Single excitation tagging}
\label{single_tagging}
The algorithm is designed to generate all $\ket {G_{pq}^{rs}}$, which are doubly excited from $\ket G$. The singly excited determinants are not explicitly generated, but are formally present as $\ket{G_{pq}^{ps}}$.
The issue is that $\ket{G_{pq}^{ps}}$ refers to the same determinant $\ordering a^\dagger_s a_q \ket{G}$ regardless of $p$, and the base algorithm only tags a $\kalpha$ as duplicate if it has a previous generator $\ket K$, i.e. if
\begin{equation}
\ket {G_{pq}^{rs}} = \ket {K_{p'q'}^{r's'}} \; ; \; \ket G = \ket {D_I} \; ; \; \ket K = \ket {D_{I'<I}}
\end{equation}
As can be seen this doesn't cover the case where $G_{pq}^{ps} = G_{p'q}^{p's}$.

To solve this issue, we default to tag $\ket {G_{pq}^{ps}}$, which prevents generating single excitations, and selectively untag in certain cases:


\begin{itemize}
\item
\textbf{Untagging all $\alpha$-spin single excitations of $\ket G$ exactly once}:

Pick $P$ any ``non-frozen'' $\beta$ spinorbital occupied in $\ket G$. We arbitrarily choose the lowest one. Untag $\ket {G_{Pq}^{Ps}}$ whenever $q,s$ are of $\alpha$ spin. Any $\alpha$-spin single excitation $q \rightarrow  s$ is untagged a single time.

$P$ cannot be chosen of $\alpha$ spin, because single excitations $P \rightarrow  s$ and $q \rightarrow  P$ would be formally present as $G_{PP}^{Ps}$ and $G_{Pq}^{PP}$, which aren't ever generated, since for obvious reasons the base algorithm never considers batch $\ket {G_{qq}}$ or determinant $\ket {G_{pq}^{rr}}$.
\item
\textbf{Untagging all $\beta$-spin single excitations of $\ket G$ exactly once}:

Pick $Q$ any ``non-frozen'' $\alpha$ spinorbital occupied in $\ket G$. Again we arbitrarily choose the lowest one. If $p,q$ are of $\beta$ spin, untag $G_{pQ}^{rQ}$. Any $\beta$-spin single excitation $p \rightarrow  r$ is untagged a single time.
\end{itemize}



\section{Systematic determination of connections}

The systematic determination of connections between $\ket S$ and determinants from the $\ket {G_{pq}}$ batch is done by comparing $\ket S$ to the doubly ionized determinant $\ket {G_{pq}}$. This yields a set of spinorbitals with different occupation status. Remembering $\ket S$ has two extra electrons compared to $\ket {G_{pq}}$, there are 4 cases of interest:
\begin{itemize}

\item
$i$,$j$ are occupied in $\ket S$ but not in $\ket {G_{pq}}$
\item
$i$,$j$,$k$ are occupied in $\ket S$ but not in $\ket {G_{pq}}$ ; $a$ is occupied in $\ket {G_{pq}}$, but not in $\ket S$
\item
$i$,$j$,$k$,$l$ are occupied in $\ket S$ but not in $\ket {G_{pq}}$ ; $a$,$b$ are occupied in $\ket {G_{pq}}$, but not in $\ket S$
\item
More differences : $\ket S$ isn't connected to any $\ket {G_{pq}^{rs}}$ and can be ignored. 

\end{itemize}

Based on these indices, it's possible to immediately deduce any $(r,s)$ pair so that $\ket {G_{pq}^{rs}}$ is at most a double excitation of $\ket S$, as well as the excitation operator $\hat{T}$ so that $\ket {G_{pq}^{rs}}=\hat{T} \ket S$. Figures \ref{fig:systematic_determination} and \ref{fig:systematic_determination2} show two possible cases as example.

While this could be done in a more compact way, we took a more case by case approach, allowing more specialized code for each situation. Taking spin into account, the different cases are listed in table \ref{tab:systematic_determination}.


It's noticeable that, because of the ``wildcard'' indices $X$ and $Y$ :
\begin{itemize}

\item
Cases of the form $a,ijk$ cause full rows/columns of $P(G_{pq})$ to be tagged or incremented.
\item
Cases of the form $ij$ cause the whole $P(G_{pq})$ matrix to be tagged or incremented. Obviously, tagging the whole matrix means stopping the computation for $\ket {G_{pq}}$.
\end{itemize}

\begin{figure}[h!]
        \begin{center}
                \includegraphics[width=0.70\columnwidth]{figures/cipsi/systematic_determination}
                \caption{Illustrative example of systematic determination of connection between a selector $\ket S$ and determinants of the $\ket {G_{pq}}$ batch when $p$ and $q$ have the same spin.
}
                \label{fig:systematic_determination}
        \end{center}
        
\end{figure}


\begin{figure}[h!]
        \begin{center}
                \includegraphics[width=0.90\columnwidth]{figures/cipsi/systematic_determination2}       
                \caption{Illustrative example of systematic determination of connection between a selector $\ket S$ and determinants of the $\ket {G_{pq}}$ batch when $p$ and $q$ are of different spin.
}
                \label{fig:systematic_determination2}
        \end{center}
\end{figure}



\begin{algorithm}
        \caption{Unfiltered CIPSI selection}
        \label{alg:selection}
        \tcc{For simplification purpose, a determinants $\ket S$ is here represented by a single bitstring $S_\sigma$ of size $2N_{orb}$ where each bit is associated with a spinorbital}
        \KwData{ $\ket \Psi$, i.e. $\{D_I\}$ the set of internal determinants and their coefficients $c_I$}
        \KwData{ $\Ngen$, $\Nsel$, $\Ndet$}
        %\KwResult{ $\Hij{\alpha}{\Psi} \neq 0$ has been computed exactly once for any $\alpha \notin \Psi$ }
        \KwResult{ $e_\alpha \neq 0$ has been computed exactly once for any $\kalpha \notin \{D_I\}$ }
                
        \For{$g \gets 1, \Ngen$}{
        \ForAll{$(p,q) \; ; \; a_p a_q \ket {D_g} \neq 0$}{
                %$\ket {G_{pq}} \gets \ordering a_p a_q \ket {D_g}$\;
                \tcc{$B$ and $P$ are indexed by spinrobitals}
                $B$ a $\mathtt{FALSE}$ initialized boolean matrix size $2N_{orb} \times 2N_{orb}$ \;
                $P$ a zero-initialized real matrix size $2N_{orb} \times 2N_{orb}$ \;
                
                apply null determinants and single excitations tagging (algorithm \ref{alg:unblock_single}) \;
                
                \For{$s \gets 1,N_{det}$}{
                        $\ket S \gets \ket {D_s}$ \; 
                        $C_\sigma \gets S_\sigma \wedge \neg [G_{pq}]_\sigma$ \; %c \gets \IAND{S}{\NOT{G_{pq}}}$ \; 
                        \If{$||C_\sigma|| = 2$}{
                                $e \gets \mathtt{LIST\_FROM\_BITSTRING}(C_\sigma)$\; 
                                $B_{e[1], e[2]} \gets \mathtt{TRUE}$\;                     
                        }                       
                        \tcc{see table \ref{tab:systematic_determination} for $(r,s)$ pairs}
                        \uIf{$s < g$}{
                                    
                                \ForAll{$(r, s) \; ; \; \Hij{S}{G_{pq}^{r s}} \neq 0$}{
                                  $B_{rs} \gets \mathtt{TRUE}$ \;
                                } 
                        }
                        \uElseIf{$s \leq \Nsel$}{
                                \ForAll{$(r, s) \; ; \; \neg B_{rs} \wedge \Hij{S}{G_{pq}^{r s}}\neq 0 $ }{
                                  $P_{rs} \gets P_{rs} + c_s \Hij{S}{G_{pq}^{rs}}$ \;
                                }
                        }
                }
                \ForAll{$(r,s) \; ; \; \neg B_{rs}$}{
                  $\kalpha = \Gpqrs$ is a unique $\kalpha$ \;
                  $e_\alpha = \frac{{P_{rs}}^2}{\Delta E_\alpha}$ \;
                  %$P_{rs}$ is $\Hij{G_{pq}^{rs}}{\Psi}$ \;
                }
        }
        } 
\end{algorithm}


\begin{algorithm}
        \caption{Null determinants and single excitations tagging}
        \label{alg:unblock_single}
        \KwData{$B$, $q$, $p$ and $\ket {G_{pq}}$ from outer scope. }
        \KwResult{Updates $B$ so as to tag null determinants, and determiants that were previously generated from a single excitation on $\ket G$}

        \tcc{tag null determinants}
                \ForAll{$r \; ; \; (a_r \ket {G_{pq}} \neq 0) \vee (r=p) \vee (r=q)$}{
                        $B_{*r} \gets \mathtt{TRUE}$ \;
                        $B_{r*} \gets \mathtt{TRUE}$ \;
                }
        \tcc{tag duplicate single excitations}
                \If{($q$ is of spin $\alpha$) $\wedge$ ($p$ is the lowest ``non-frozen'' $\beta$ spinorbital)}{
                                $B_{*p} \gets \mathtt{FALSE}$ \;  
                                $B_{p*} \gets \mathtt{FALSE}$ \;          
                        }
                        
                        \If{($p$ is of spin $\beta$) $\wedge$ ($q$ is the lowest ``non-frozen'' $\alpha$ spinorbital)}{
                                $B_{*q} \gets \mathtt{FALSE}$ \; 
                                $B_{q*} \gets \mathtt{FALSE}$ \;         
                        }
\end{algorithm}



\section{Filtering and loop breaking}

A large amount of processing power is lost, because every doubly ionized generator $\ket {G_{pq}}$ is compared to all internal determinants. In the vast majority of cases, it will show no connection can be made and the internal determinant will be ignored. Thus, it's interesting to filter internal determinants in more outer loops (loop over generators, and loop over first ionization).

This can be done using the distance $f_A^B = f_B^A$, defined as the minimal number of operations --- moving, annihilating or creating an electron --- that must be done to go from a determinant $\ket A$ to a determinant $\pm \ket B$ (i.e. ignoring the phase factor) with respectively $n_a$ and $n_b$ electrons.
Alternatively, it can be defined as the maximum between the number of annihilations and the number of creations required to go from $\ket A$ to $\pm \ket B$.




\begin{equation}
f_A^B = \frac{||A_\alpha \oplus B_\alpha|| + ||A_\beta \oplus B_\beta|| + |n_a-n_b|}{2}
\end{equation}


Considering $\ket S$ a selector determinant and $\ket X$ a generator determinant in a state of ionization from 0 to 2 (it essentially is a wildcard for $\ket G$, $\an p \ket G$ or $\an p \an q \ket G = \ket {G_{pq}}$).

\begin{itemize}
\item
$f_X^\alpha + f_\alpha^S \geq f_X^S$
\item
$\ket \alpha$ can be generated from $\ket X$ iff $f_X^\alpha \leq 2$
\item
$\ket \alpha$ is connected to $\ket S$ iff $f_\alpha^S \leq 2$
\item
$0 \leq (f_Y^S - f_X^S) \leq 1$ with $\ket Y = a_p \ket X$
\end{itemize}

\alert{à detailler?}

From the rules above, we can deduce that given any $\ket X$ and $\ket S$, there exists an $\kalpha$ generated from $\ket X$ so that $\Hij{\alpha}{S} \neq 0$ only if $f_X^S \leq 4$.
Based on this, a filtering system can be set up, as shown on figure \ref{fig:selection}. The diagram is somewhat convoluted and deserves comments.

\paragraph{Internal determinants' path}
 
A triple loop is shown

\begin{enumerate}
\item
over generators ($G$)
\item
over $p$ a first ionization ($G_p$)
\item
over $q$ a second ionization ($G_{pq}$), i.e. over batches.
\end{enumerate}

In each one some filtering takes place. The determinants of the internal space ``flow'' from the top $\Psi$ into intermediate lists, that are fully constructed before proceeding to the inner loop, as they will be the sources of determinants for that inner loop.
A selector can only be duplicated at the node denoted by a black circle. Otherwise, it follows a single path, always going for the horizontal path if it satisfies the associated condition.
If it doesn't satisfy the condition of an horizontal path, and there is no further vertical path, it is discarded.

\paragraph{``Drop'' instructions}
\emph{Drop} instructions are reached when, predictably, the current loop iteration will not yield any unique $\kalpha$. If a determinant reaches a \emph{drop}, the current loop iteration ends immediately.

\begin{itemize}
\item
\emph{drop} $G_{pq}$ is reached in the case where the whole $P(G_{pq})$ matrix is to be tagged, i.e. the possible values for $(r,s)$ given by table \ref{tab:systematic_determination} are two wildcards ($X,Y$ and $X,\bar Y$).
This corresponds to the case where $\ket {G_{pq}}$ has already been created from a previous generator $\ket K$, i.e. $\ket {G_{pq}} = \ket {K_{p'q'}}$, therefore for any pair $(r,s)$ we have $\ket {K_{p'q'}^{rs}} = \ket {G_{pq}^{rs}}$.\\
\item
\emph{drop} $G_{p}$, in the same fashion, is reached when $\ket {G_{p}}$ has already been created from a previous generator $\ket K$, i.e. $\ket {G_{p}} = \ket{K_{p'}}$. For any triplet $(q,r,s)$ there will be $\ket {K_{p'q}^{rs}} = \ket {G_{pq}^{rs}}$, so no new $\kalpha$ will be created.
\end{itemize}


\paragraph{Paths and loops}
There are roughly a left and a right path. The reason for this, is that we want to reach \emph{drop} instructions as fast as possible. Incidentally, in each loop, the implementation should prioritize operations that may cause a reach to \emph{drop}.

%Not trying to reach \emph{drop} $G_{PQ}$ means we might completely compute the $P$ matrix, only to find that the last selector determinant tags it entierly, for the reson mentioned above.
%Not trying to reach \emph{drop} $G_{P}$ means we might iterate over batches $\Gpqrs$ that are all to be "tagged out" by a single particular selector.

\begin{enumerate}

\item
The first loop discards some internal determinants and separates the others in two disjoint categories.


\begin{itemize}
\item
Right branch : determinants that may contribute to some $P(G_{pq})$ matrix or tag previously generated $\ket \alpha$. In other words, selectors that may connect to some $\ket {G_{pq}^{rs}}$. 

\item
Left branch : Determinants that aren't selectors, but are equal to some $\ket {G_{pq}^{rs}}$. Being non-selectors, those will not be checked for connection to any $\ket {G_{pq}^{rs}}$, but they still must be checked for equality in order to ensure $\ket {G_{pq}^{rs}} \notin \Psi$
\end{itemize}

This step sets the algorithm's complexity with respect to $\Ndet$. Naively, $f_G^S$ must be computed for all pairs of internal determinants, setting the complexity to $\mathcal{O}(\Ndet^2)$. Our current implementation quickly discards $f_G^S > 4$ by using a method similar to what we used in the Davidson diagonalization, adapted to seek excitation degrees $\leq 4$ rather than $\leq 2$. The key difference is that, for parallelism reasons, the research has to be done individually for each generator ; that is, we are not computing all $f_G^S$ at the same time, but all $f_G^S$ for a given $G$ separately. The procedure is shown as algorithm \ref{alg:generators_filtering}. The complexity is reduced from $\mathcal{O}(\Ndet^2)$ to $\mathcal{O}(\Ndet^{3/2})$ ; each generator has its $\sigma \in \{\alpha, \beta\}$ part compared to all elements of $\{U_\sigma\}$.


\begin{algorithm}
\caption{Filtering internal determinants for generator $\ket G$}
\label{alg:generators_filtering}

\For{$S_\alpha \in \{U_\alpha \}$}{
$a \gets \text{EXC\_DEGREE}(S_\alpha, G_\alpha)$ \;
\If{$a \leq 2$}{
        find $S_\beta \in \{U_\beta\}_{G_\alpha}$ so that $\text{EXC\_DEGREE}(S_\beta, G_\beta) + a \leq 4$  \; 
}
}
\For{$S_\beta \in \{U_\beta \}$}{
$b \gets \text{EXC\_DEGREE}(S_\beta, G_\beta)$ \;
\If{$b \leq 1$}{
        find $S_\alpha \in \{U_\alpha\}_{G_\beta}$ so that $\text{EXC\_DEGREE}(S_\alpha, G_\alpha) + b \leq 4$  \; 
}
}
\end{algorithm}



Note that the only point of separating those two categories rather than merging them in the same list, is to avoid additional $past$ and $selector$ tests in the second loop.
%If both those lists were merged, a $selector$ condition would need to be added for reaching the right list of the second loop, and $past$ tests would be performed pointlessly on those determinants that would have gone in the left list of the first loop.
This most likely is of little interest, depending on the implementation.
%$past$ and $selector$ should at worst mean fetching an index in an array of indices, and compare it to $I$ or $N_{sel}$ respectively.
But because it's the actual implementation and because it reduces the number of operations, it is still shown.

\item
The second loop discards some internal determinants and separates the other in two categories, this time not disjoint.

\begin{itemize}
\item
Right branch : Determinants that may contribute to some $\Hij{\Psi}{\alpha}$, i.e. selectors that may connect to some $\ket {G_{pq}^{rs}}$
\item
Left branch : Determinants that may cause tagging or reach \emph{drop}, i.e. any determinant that may be equal to some $\ket {G_{pq}^{rs}}$, and previous generators that may connect to some $\ket {G_{pq}^{rs}}$. Those can be found in both lists built in the first loop.
\end{itemize}

As explained above, if there is a previous generator $\ket K$ so that $a_{p'} \ket K = a_p \ket G$, it will result in $P(G_{pq})$ being fully tagged for any $q$, hence a need to reach \emph{drop} $G_p$ to avoid unnecessary computations.
The reach for \emph{drop} $G_p$ can be put on the path between the right list of the first loop and the left list of the second loop.

Indeed, $a_{p'} \ket K = a_p \ket G$ with $\ket K$ a previous generator translates to
\begin{equation}
(f^K_{G_{p}} = 1) \wedge past
\end{equation}
The right list of the first loop contains all internal determinants so that
\begin{equation}
(f^K_G \leq 4) \wedge selector
\end{equation}
However 
\begin{equation}
f^K_{G_{p}} = 1 \implies f^K_G \leq 1 \implies f^K_G \leq 4
\end{equation}
\begin{equation}
past \implies selector
\end{equation}
\begin{equation}
(f^K_{G_{p}} = 1) \wedge past \implies (f^K_G \leq 4) \wedge selector
\end{equation}

Therefore any internal determinant able to reach \emph{drop} $G_P$ will be present in that list. Trivially, from there it will always take the left path because $f^K_{G_{p}} = 1 \implies f^K_{G_{p}} \leq 2$.


\item
Third loop :
\begin{itemize}

\item
Right branch :
Final filtering to keep only selectors that do connect to some $\Gpqrs$
\item
Left branch : $f_{G_{pq}}^S = 2$ implies there exist $(r,s)$ so that $\ket S=\Gpqrs$. If it is the case:
\begin{itemize}
\item
If $past$
\begin{equation}
\ket S=\ket {G_{pq}^{rs}} \implies \ket {S_{rs}} = \ket {G_{pq}}
\end{equation}
As explained above, it leads to $P(G_{pq})$ being fully tagged, and thus \emph{drop} $G_{pq}$ can be reached.
\item
If $\neg past$, $\ket {G_{pq}^{rs}}$ must be tagged for referring to a determinant of the internal space.
\end{itemize}



\end{itemize}

\end{enumerate}



\newcommand{\Gpq}{\ket {G_{pq}}}
\newcommand{\Gpbq}{\ket {G_{p \bar q}}}

\begin{table}

\caption{Systematic ``case by case'' determination of connections between a selector $\ket S$ and determinants of a batch $\Gpq$} 
\label{tab:systematic_determination}
\begin{center}
        \begin{tabular}{ c|c|c }
                \hline \hline \rule{0pt}{3ex}
                $\ket S$                                                                        &$ r, s$        & $\hat T ; \hat T \ket S = \Gpqrs$     \\
                \hline \hline \rule{0pt}{3ex}
                $\ac {ij} \Gpq$                                         & $X,Y$         &$ij \rightarrow XY$            \\
                                                                                        & $X,i$         &$j \rightarrow X$              \\
                                                                                        & $i,j$         &$1$                    \\
                \hline \rule{0pt}{3ex}
                $\an a \ac {ijk} \Gpq$                          &$X,i$          &$aX \rightarrow jk$            \\
                                                                                        &$i,j$          &$a \rightarrow k$              \\
                \hline \rule{0pt}{3ex}
                $\an {\bar a} \ac {\bar i jk} \Gpq$     &$X,j$          &$\bar a k \rightarrow \bar i X$                \\
                                                                                        &$j,k$          &$\bar a \rightarrow \bar i$            \\
                \hline \rule{0pt}{3ex}
                $\an {ab} \ac {ijkl} \Gpq$                      &$i,j$          &$ab \rightarrow kl$            \\
                \hline \rule{0pt}{3ex}
                $\an {a  \bar b} \ac {ijk \bar l} \Gpq$                 &$i,j$          &$a \bar b \rightarrow k \bar l$                \\
                \hline \rule{0pt}{3ex}
                $\an {\bar a \bar b} \ac {i j \bar k \bar l} \Gpq$      &$i,j$          &$\bar a \bar b \rightarrow \bar k \bar l$              \\
                
                \hline \hline \rule{0pt}{3ex}
                $\ket S$                                                                        &$ r, \bar s$   & $\hat T ; \hat T \ket S = \ket {G_{p \bar q}^{r \bar s}}$     \\
                \hline \hline \rule{0pt}{3ex}
                $\ac {i \bar j} \Gpbq$                          & $X, \bar Y$   &$i \bar j \rightarrow X \bar Y$                \\
                                                                                        & $i,\bar X$            &$\bar j \rightarrow \bar X$            \\
                                                                                        & $X,\bar j$    &$i \rightarrow X$              \\
                                                                                        & $i,\bar j$    &$1$                    \\
                                                                                        
                                                                                        
                \hline \rule{0pt}{3ex}
                $\an a \ac {ij \bar k} \Gpbq$           &$X,\bar k$             &$aX \rightarrow ij$            \\
                                                                                        &$i,\bar k$             &$a \rightarrow j$              \\
                                                                                        &$i,\bar X$             &$a \bar k \rightarrow j \bar X$                \\
                                                                                        
                \hline \rule{0pt}{3ex}
                $\an {ab} \ac {ijk \bar l} \Gpbq$                       &$i,\bar l$             &$ab \rightarrow jk$            \\
                \hline \rule{0pt}{3ex}
                $\an {a  \bar b} \ac {ijk \bar l} \Gpbq$                        &$i,j$          &$a \bar b \rightarrow k \bar l$                \\
                \hline \rule{0pt}{3ex}
                $\an {a  \bar b} \ac {ij \bar k \bar l} \Gpbq$                  &$i,\bar k$             &$a \bar b \rightarrow j \bar l$                \\
        \end{tabular}
        
\end{center}
\begin{itemize}
\item
\textbf{the bar notation $\bar a$ is used to indicate relative spins}
%, i.e. $ab$ means $a$ and $b$ are of same spin, $a\bar b$ means they are of different spin.
\item
$\an{ij\ldots}$ is a compact notation for $\an i \an j \ldots$
\item
$X$ and $Y$ are "wildcard" indices referring to any spinorbital unoccupied in both $\ket S$ and $\ket {G_{pq}}$ 
\end{itemize}
%- "is in wavefunction" : $\ket {G_{pq}^{rs}} = S$. There is no need to compute $\langle \ket {G_{pq}^{rs}}|H|S \rangle$ since $\ket {G_{pq}^{rs}}$ is necessarily tagged for being present in the wavefunction.
\end{table}


\begin{figure}[h!]
        \begin{center}
                \includegraphics[height=0.90\textheight]{figures/cipsi/selection}
        \end{center}
        \caption{$S$ is the internal determinant currently flowing down the chart.
        Tagging is fully computed, and \emph{drop} instructions eventually reached before any update is done to $P(G_{pq})$.}
        \label{fig:selection}   
\end{figure}

\section{Parallel computation}

Arguably the simplest way to make an algorithm parallel is, whenever possible, to create independent tasks corresponding to one iteration of the outermost loop. 

As figure \ref{fig:selection} suggests, iterations for the outermost loop --- over generators --- are independent. This is due to our choice to perform the initial filtering on a ``generator by generator'' basis, giving a total complexity of $\mathcal{O}(\Ndet^{3/2})$ (algorithm \ref{alg:generators_filtering}) rather that $\mathcal{O}(\Ndet)$ as our Davidson algorithm does ; to achieve $\mathcal{O}(\Ndet)$ the outermost loop would need to be over $\{U_\alpha\}$ and $\{U_\beta\}$. This would give a smaller number of iterations/tasks with extremely unbalanced costs, which doesn't suit well a parallel scheme. Since in practice the filtering steps only account for a few percent of the total CPU time, we stuck to 1 task = 1 generator. Even so, the cost for different tasks was still very much unbalanced, the first few (generators of large coefficients) being very expensive, and the cost quickly decreasing.

For a better load balancing, we split the first, expensive tasks into smaller \emph{fragments}, using a fairly simple approach. Essentially, some tasks will require computing just a subset of the $\ket {G_{pq}}$ batches associated with a generator $\ket G$, as opposed to all of them. This implies some overhead, since some filtering steps will be duplicated. However, this overhead was found to be small, and only a relatively small number of expensive tasks need to be split.



Each generator $\ket {G_i}$ defines as a ``logical'' task $i$, and for each one we define $F_i$ a fragmentation level, defining the number of independent ``actual'' tasks it should be decomposed in (in practice $F_i = 1$ for the majority). Ideally, $F_i$ should be based on an estimated cost of task $i$. 

Task $i$ is put in the job queue $F_i$ times, each time associated with an index $s$ ranging from $0$ to $F_i-1$, which together with $F_i$ defines the subset of batches this task corresponds to. Disjoint subsets are created using modulus, so that subsets are interleaved and thus more balanced.
 
This fragmentation scheme can be shown in a simpler and more general way when used to compute $\EPT$ in chapter \ref{chap:PT2}. In a nutshell, we can see it as a toy problem where we want to print for each generator the sum of $\epsilon(\alpha)$ over all unique $\kalpha$ it has generated. The algorithm for this on the master and slave side is shown as algorithm \ref{alg:tasksplit_master} and \ref{alg:tasksplit_slave} respectively.


\begin{algorithm}
        \label{alg:tasksplit_master}
        \caption{task splitting, pseudocode for master}
        \tcc{A logical task is the computation of $e[i]$, the sum of $\epsilon(\alpha)$ over all unique $\kalpha$ generated from generator $\ket {G_i}$}
        choose $F_i$ \;
        \tcc{A logical task is the computation of $e[i]$, the sum of $\epsilon(\alpha)$ over all unique $\kalpha$ generated from generator $\ket {G_i}$}
        choose $F_i$ for all $i$\;
        \For{$i=1,\Ngen$}{
        \For{$s=0,F_i-1$}{
                add task $(i,s)$ to the queue   
        }
        }
        $f$,$e$ are arrays size $\Ngen$ initialized with $0$ \;
        \While{not all $e[i]$ printed}{
                get $(i,\text{sum})$ from a slave \;
                $e[i] \gets e[i] + \text{sum}$ \;
                $f[i] \gets f[i] + 1$ \;
                \If{$f[i] = F_i$}{
                        print $e[i]$ \;
                }
        }
\end{algorithm}

\begin{algorithm}
        \label{alg:tasksplit_slave}
        \caption{task splitting, pseudocode for slave}
        \While{}{
                get task $(i, s \in [0,F_i-1])$ from the queue \;
                $E \gets 0$ \;
                $c \gets 0$ \;
                $G \gets D_I$ \;
                \tcc{duplicated computation if $F_i > 1$}
                filtering for $G$ (see figure \ref{fig:selection}) \;
                \ForEach{$G_p$}{
                \tcc{duplicated computation if $F_i > 1$}
                filtering for $G_p$ (see figure \ref{fig:selection}) \;
                \ForEach{$G_{pq}$}{
                        $c \gets c + 1$ \;
                        \If{$s = c \mod F_i$}{
                                increment $E$ with all unique $\epsilon(\alpha)$ in this batch \;
                        }
                }
                }
                send $(i, E)$ to master \;
        }
\end{algorithm}


\alert{Il faut mettre une petite conclusion / resume ici.}

\end{document}
