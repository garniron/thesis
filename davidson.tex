\documentclass[./thesis.tex]{subfiles}

 
\begin{document}

The diagonalization of the Hamiltonian is a necessary step in configuration
interaction. Standard diagonalization algorithms scale as $\order{\Ndet^3}$ in
terms of computation, and $\order{\Ndet^2}$ in terms of storage, so the cost is
prohibitive as $\Ndet$ ranges usually between a few million and a few billion.

Fortunately,  not all the spectrum of $\hat H$ is required: only the few lowest
eigenstates are of interest. The Davidson algorithm\cite{Davidson_1975} is an
iterative algorithm which aims at extracting the first few $\Nstate$ lowest
eigenstates of a large matrix. This algorithm reduces the cost of the 
computation and the storage to $\order{\Nstate \Ndet^2}$, and is presented as
algorithm~\ref{alg:davidson}.


\begin{algorithm}
 \newcommand{\mU}{\mathbf{U}}
 \newcommand{\mW}{\mathbf{W}}
 \newcommand{\mR}{\mathbf{R}}
 \newcommand{\mh}{\mathbf{h}}
 \newcommand{\my}{\mathbf{y}}
 \caption{Davidson's diagonalization algorithm}
 \label{alg:davidson}
	\SetKwFunction{FMain}{DAVIDSON\_DIAG}
	\SetKwProg{Fn}{Function}{:}{}
\Fn(){\FMain{$\Nstate, \mU$}}{
	\KwData{ $\Nstate$~: Number of requested states}
	\KwData{ $\Ndet$~: Number of determinants}
	\KwData{ $\mU$~: Guess vectors, $\Ndet \times \Nstate$}
	\KwResult{ $\Nstate$ lowest eigenvalues eigenvectors of $\mH$ }
\texttt{converged} $\gets \FALSE$ \;
\While{$\neg{\texttt{converged}}$}
{
  Gram-Schmidt orthonormalization of $\mU$ \;
  $\mW \gets \mH\, \mU$ \; 
  $\mh \gets \mU^\dagger\, \mW$ \;
  Diagonalize $\mh$ : eigenvalues $E$ and eigenvectors $\my$ \;
  $\mU' \gets \mU\, \my$ \;
  $\mW' \gets \mW\, \my$ \;
  \For{$k=1,\Nstate$}{
    \For{$i=1,\Ndet$} {
      $\mR_{ik}  \gets \frac{E_k \mU_{ik}' - \mW_{ik}'}{\mH_{ii} - E_k}$ \;
    }
  }
  $\texttt{converged} \gets \norm{\mR} < \epsilon$ \;
  $\mU \gets [ \mU, \mR ]$ \;
}
\KwRet{$\mU$}\;
}
\end{algorithm}



\alert{
\begin{itemize}
\item Parler de quelques ameliorations depuis : [ J.  Olsen, P.  Joregensen and J.  Simons, Chem.  Phys.  Letters 169 (1990) 463 ], [F.X.  Gadea /Chemical Physics Letters 227 (1994) 201-210] 
\item Dire qu'on utilise l'algorithme original parce qu'on a toujours un bon guess (CIPSI)
\item Pour etre fonction propre de $S^2$ on utilise une penalty method, comme dans [10.1021/acs.jctc.7b00466]. (D'ailleurs dans ce papier il y a un pseudocode pour Davidson). Et tu peux ajouter qu'on calcule $S^2c$ en meme temps que $Hc$.
\end{itemize}
}

    
Algorithmically, the expensive part of the Davidson diagonalization is the computation of the matrix product $\mH\, \mathbf{U}$.
Determinants are connected by $\mH$ only if they differ by no more than two
spinorbitals. Therefore, the number of non-zero elements per line of $\mH$ is
equal to the number of single and double excitation operators, namely
$\order{\Nelec^2 \times (\Norb - \Nelec)^2}$. As $\mH$ is symmetric, the number
of non-zero elements per column is identical. This makes the $\mH$ matrix very
sparse, but for large basis sets the whole $\mH$ matrix may not fit in the
memory of a single node, as the number of non-zero entries to store is
$\order{\Ndet \times \Nelec^2 \times (\Norb - \Nelec)^2}$.  One possibility
would be to distribute the storage of Hamiltonian among multiple compute nodes,
and use a distributed library such as PBLAS\cite{pblas} to perform the
matrix-vector operations. Another approach is to use a so-called \emph{direct}
algorithm, where the matrix elements are computed \emph{on the fly}, and this
is the approach chosen in the \QP.


This effectively means iterating over all pairs of determinants $\ki$ and
$\kj$, checking whether $\ki$ and $\kj$ are connected by $\mH$ and if they are,
accessing the corresponding integral(s) and computing the phase factor.  Even
though we presented a very effective method to compute the excitation degree
between two determinants, the number of such computations to be made scales as
$\Ndet^2$, which can still be prohibitively high. To get an efficient
implementation it is mandatory to filter out all pairs of determinants that are
not connected by $\mH$, and iterate only over connected pairs.  We have
implemented an algorithm similar to the \emph{Direct Selected Configuration
Interaction Using Strings} (DISCIUS) algorithm,\cite{Povill_1995}.

We present our algorithm for finding non-null $\widehat{H}$ elements - in other words, pairs of connected internal determinants - as algorithm ~\ref{alg:davidson_aa} and \ref{alg:davidson_ab}. The finality is to build \alert{deja un autre nom?} $W = \widehat{H} c$, so each time a connected pair of determinants $(\ket I, \ket J \neq \ket I)$ is found, $W$ should be updated accordingly

\begin{align}
W_I \gets W_I + c_J \Hij{I}{J} \\
W_J \gets W_J + c_I \Hij{I}{J}
\end{align}

Note that diagonal elements are not found be either algorithm.


\begin{algorithm}
\caption{Find internal determinants connected by purely $\alpha$ or purely $\beta$ single or double excitations}
\label{alg:davidson_aa}
\KwData{$N_a$ the number of unique $\alpha$ spin parts present in $\ket \Psi$}
\KwData{$D$ the array of determinants present in $\ket \Psi$, sorted by alpha-major order (all determinants sharing the same $\alpha$ part are next to each other)}
\KwData{$A$ the array so that $A[n]$ is the index of the first occurence of the $n^{th}$ unique $\alpha$ spin part in $D$. For algorithmic convenience we set $A[N_a +1] = \Ndet+1$}

\For{$a \gets 1, N_a$}{
\tcc{All determinants sharing $D[A(a)]_\alpha$ alpha spin part are in the range $[A(a), A(a+1)-1]$}
\For{$b1 \gets A(a), A(a+1)-1$}{
\For{$b2 \gets b1+1, A(a+1)-1$}{
	\If{$\text{EXC\_DEGREE}(D[b1]_\beta, D[b2]_\beta) \leq 2$}{
		$\ket {D[b1]}$ connected to $\ket {D[b2]}$ by simple or double $\beta$ excitation.	
	}
}
}
}
\tcc{simple and double $\alpha$ excitations are found by the same algorithm after flipping spins}
\end{algorithm}


\begin{algorithm}
\label{alg:davidson_ab}
\caption{Find internal determinants connected by $\alpha \beta$ double excitations}
\KwData{see algorithm \ref{alg:davidson_aa}}

\For{$a1 \gets 1, N_a$}{
\For{$a2 \gets a1+1, N_a$}{
\If{$\text{EXC\_DEGREE}(D[A(a1)]_\alpha, D[A(a2)]_\alpha) \neq 1$}{
	cycle $a2$ loop\;
}
\For{$b1 \gets A(a1), A(a1+1)-1$}{
\For{$b2 \gets A(a2), A(a2+1)-1$}{
	\If{$\text{EXC\_DEGREE}(D[b1]_\beta, D[b2]_\beta) = 1$}{
		$\ket {D[b1]}$ connected to $\ket {D[b2]}$ by $\alpha \beta$ excitation.	
	}
}
}
}
}
\end{algorithm}

\begin{algorithm}
\label{alg:davidson_ab_parallel}
\caption{Find internal determinants connected by $\alpha \beta$ double excitations, one in the range $[\text{first}, \text{last}]$ the other in the range $[\text{first}, \Ndet]$}
\KwData{see algorithm \ref{alg:davidson_ab}}
\KwData{$\text{first}$, $\text{last}$ : the boundaries of the range of determinants (in $D$) the current task should process}

\For{$a1 \gets 1, N_a$}{
\If{$A(a1+1) -1 < \text{first}$}{
	cycle $a1$ loop\;
}
\If{$A(a1) > \text{last}$)}{
	return \;
}
$f \gets max(\text{first}, A(a1))$ \;
$t \gets min(\text{last}, A(a1+1)-1)$ \;
\For{$a2 \gets a1+1, N_a$}{
\If{$\text{EXC\_DEGREE}(D[A(a1)]_\alpha, D[A(a2)]_\alpha) \neq 1$}{
	cycle $a2$ loop\;
}
\For{$b1 \gets f,t$}{
\For{$b2 \gets A(a2), A(a2+1)-1$}{
	\If{$\text{EXC\_DEGREE}(D[b1]_\beta, D[b2]_\beta) = 1$}{
		$\ket {D[b1]}$ connected to $\ket {D[b2]}$ by $\alpha \beta$ excitation.	
	}
}
}
}
}
\end{algorithm}

The loops finding purely $\alpha$ and purely $\beta$ excitations are considerably less costly than the one finding $\alpha \beta$ excitations, to the point of being negligible. For parallelization purpose, we only focus on $\alpha \beta$ excitations.

To separate this computation in independent tasks with no overhead, a task should corresond to one unique $\alpha$ spinpart / one value of $a1$ in algorithm \ref{alg:davidson_aa}. However, this gives a potentially small number of extremely unbalanced tasks, some unique $\alpha$ spin parts being both well-connected to other $\alpha$ parts and associated with a great number of $\beta$ parts.
Some overhead appeared unavoidable, so it was decided to associate a task with a range of internal determinants sorted in alpha-major order. This implies splitting some sets of unique alpha part, and thus having some redundance in the comparaison of the associated $\alpha$ spin part with the other unique ones.
This is achieved by small modifications to algorithm \ref{alg:davidson_ab} show in algorithm \ref{alg:davidson_ab_parallel}.


\end{document}
