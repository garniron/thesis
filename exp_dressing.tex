\documentclass[./thesis.tex]{subfiles}

 
\begin{document}



While CI methods are common ways to account for electron correlation, they suffer a severe size consistency problem.


The logics of the electronic Many-Body problem has been clarified a long time
ago in the situations where  the  wave  function  may  be  generated  from  a
single  determinant  (or  single  reference).  Perturbative  developments,
translated  in  terms  of  diagrams,  led  to  the  formulation  of  the
fundamental  linked  cluster  theorem,\cite{Goldstone}  and  clarified  the  defects  of
truncated  Configuration Interaction  methods.  The  conditions  for  a  good
scaling  of  the  correlation  energy  and  for  the  strict separability  into
closed  shell  fragments  were  established.

By  strict  separability  (which
is  less ambiguous than the terms size-extensivity and size consistency) we
mean that at the non-interacting limit of an $A\cdots B$ problem the energies are
additive, $E_{A\cdots B} = E_A+E_B$, and that the amplitudes associated with the single
and  double excitation operators  are  the  same  as  those  obtained  for  the  isolated  $A$
and  $B$  problems.

Why this is not the case in CI methods can be easily understood as the absence of some excitations simulatenously on all subsystems.

Considering a supersystem $C$ made of two non-interacting subsystems $A$ and $B$, and write its CID wavefunction (Hartree-Fock determinant and all its double excitations).

\begin{equation}
\Psi^C = \Psi_{HF}^C + \Psi_{D}^C
\end{equation}

with $\Psi_{HF}^C$ the Hartree-Fock determinant for system $C$, and $\Psi_{D}^C$ the sum of all double excitation with respect to $\Psi_{HF}^C$.

If we now write $\Psi^{A+B}$ the product of the two separate CID wavefunctions of $A$ and $B$


\begin{align}
\Psi^{A+B} = & \Psi^A  \Psi^B \\
 = & \Psi_{HF}^A\Psi_{HF}^B  + \Psi_{HF}^A\Psi_{D}^B + \Psi_{D}^A\Psi_{HF}^B + \Psi_{D}^A \Psi_{D}^B \\
 = & \Psi^C_{HF} + \Psi^C_D + \Psi_{D}^A \Psi_{D}^B
\end{align}

As can be seen, $C$ isn't described as the product of $A+B$ as it should, since simultaneous double excitations on $A$ and $B$ cannot be accounted for

\begin{equation}
\Psi^{A+B} - \Psi^C  = \Psi_{D}^A \Psi_{D}^B
\end{equation}

Some methods aim at partially or fully correct this size-consistency error by eliminating the unlinked effects of the CISD. The so-called \emph{Davidson corrections}, that are essentially correction to the energy\cite{Langhoff_1974}. \alert{citation a verif}, and the so-called \emph{Coulpled Electron Pair Approximations (CEPA)} that correct the CI equations\cite{Kelly_1963,Kelly_1964,Meyer_1971,Meyer_1973,Meyer_1974,Ahlrichs_1975}. \alert{grosse biblio pour un truc dont on parle pas?...}. Many different variations have been proposed, for a review see \cite{Koch_1981}.






\begin{algorithm}
\KwData{$\kalpha$ the considered exernal determinant}
\KwData{$D_i$ the list of $N$ internal determinants connected to $\kalpha$}
\KwResult{finds all diamonds in $\mathcal{O}(N_{ref} \times N)$} 

\ForAll{$\kI$ with $EXC(\kI, \kalpha) \leq 4$}{

$\delta \gets -(I + \alpha)$ \;
$i \gets 1$ \;
$j \gets 1$ \;
\While{$j \leq N \wedge i \leq N$}{
	\uIf{$D_j - D_i > \delta$}{
		increment $i$ \;
	}
	\uElseIf{$D_j - D_i < \delta$}{
		increment $j$ \;
	}
	\Else
	{
		\tcc{$\hat T_{I \rightarrow D_j}$ is $D_j-I$}
		\tcc{$D_i + (D_j-I) = \alpha$}
		
		
		
		%\tcc{\alert{check diamond}}
		%\If{$(I \oplus D_i \wedge I = D_j \oplus \alpha \wedge D_j) \wedge (I \oplus D_i \wedge D_i = D_j \oplus \alpha \wedge \alpha)$}{
		%\If{$\hat T_{I \rightarrow D_i} = \hat T_{D_j \rightarrow \alpha}$}{
		
		\tcc{\alert{check if addition applies correct excitaion}}
		\tcc{impossible exciation results in an abnormal number of modified spinorbitals}	
		\If{$||D_j \oplus I|| = ||D_i \oplus \alpha||$}{
			diamond found \;
		}
		increment $i$ and $j$ \;
	}
}
}
\end{algorithm}

Applying an impossible excitation by addition, will usually lead to an abnormal result with a wrong number of ``electrons''. But in some cases, it may lead to an existing determinant. 



This algorithm works when bitstrings are considered binary integers of arbitrary size. However, no assumption is made about the operators involved in the diamond, they can be any combination of creations and anihilations. Therefore, for simplification purpose, it is possible to consider all 64-bit integers independently, each one carrying a subset of the operators involved in the excitation.

Essentially, adding and substracting can be done integer-wise, without the added complexity of carry and overflow.
Because integers of lower $\Nint$ vary more, they should be given more weight when comparing, as this will allow to shorcut the loop over $\Nint$ faster.


\end{document}



